\documentclass{article}

\usepackage{cmap}
\usepackage[utf8x]{inputenc}
\usepackage[english,russian]{babel}

\usepackage{indentfirst}

\usepackage{listings}
\usepackage{color}

\definecolor{custom-green}{rgb}{0,0.6,0}
\definecolor{gray}{rgb}{0.5,0.5,0.5}

\lstset{frame=tb,
  language=Python,
  aboveskip=3mm,
  belowskip=3mm,
  showstringspaces=false,
  columns=flexible,
  basicstyle={\small\ttfamily},
  numbers=none,
  numberstyle=\tiny\color{blue},
  keywordstyle=\color{blue},
  commentstyle=\color{gray},
  stringstyle=\color{custom-green},
  breaklines=true,
  breakatwhitespace=true,
  tabsize=3
}

\usepackage[unicode, pdftex]{hyperref}

\begin{document}

	\begin{titlepage}
		Обзор избранных уязвимостей в безопасности python программ.
	\end{titlepage}

	\section{Введение.}
		\subsection{Вступление.}
			...
		\subsection{Постановка проблемы (что хотел сказать автор, какую проблему он затрагивает).}
            ...
		\subsection{Обрисовка проблемы, её проблемы и актуальность).}
            ...
		\subsection{Обрисовка проблемы, её аспекты и актуальность.}
            ...

	\newpage

	\section{Основная часть.}
		\subsection{[CWE-78].}
			...

		\subsection{[CWE-377]}
			...

		\subsection{Избыточные права доступа для критического ресурса[CWE-732].}
			Если ресурс имеет права доступа больше, чем нужно для нормального исполнения программы, то это может привести к раскрытию конфиденциальной информации или несанкционированному изменению этого ресурса. Это особенно опасно, когда ресурс связан с конфигурацией программы, ее выполнением или конфиденциальными данными пользователей.
			\subsubsection{Пример 1.}\label{example-1}
				Приведённый ниже код устанавливает маску режима создания пользовательских файлов (umask) процесса равной нулю, создаёт файл и записывает в него строку ``Hello, world!''.

				\begin{lstlisting}
# umask.py
import os


os.umask(0)

with open('hello.out', 'w') as f:
	f.write('Hello, world!');
				\end{lstlisting}

				\par После его исполнения на UNIX системе, результат использования команды `ls -l' может быть следующим:

				\begin{lstlisting}
-rw-rw-rw- 1 <name> <name> 13 Sep 22 11:39 hello.out
				\end{lstlisting}

				\par Строка ``rw-rw-rw-'' указывет на то, что владелец, группа и все пользователи могут читать и редактировать этот файл.

			\subsubsection{Пример 2.}
				Рассмотрим стандартный процесс создания файла.

				\begin{lstlisting}
# without-chmod.py
with open('hello.out', 'w') as f:
    f.write('Hello, world!');
                \end{lstlisting}

				\par Файл `hello.out' будет иметь параметры доступа ``rw-rw-r--''. Это значит, что сторонние пользователи не могут его редактировать.

				\par Теперь добавим несколько дополнительных строк кода.

				\begin{lstlisting}
# with-chmod.py
with open('hello.out', 'w') as f:
    f.write('Hello, world!');

from os import chmod
chmod('hello.out', 0o666)
                \end{lstlisting}

				\par После исполнения данного кода на UNIX системе, результат команды `ls -l' будет другим: ``rw-rw-rw-''. Это приводит нас к той же проблеме, что и в \ref{example-1}.

\end{document}
